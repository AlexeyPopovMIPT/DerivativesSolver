\documentclass{article}
\usepackage[russian]{babel}
\title{ Лабораторная работа "Вычисление производных элементарных функций" }
\author{ Попов Алексей }
\begin{document}
\maketitle
\section{ С гордостью представляю вам: }
Функция!
\begin{center}$({x^{{2}}}+sinh(a \cdot x \cdot x+b \cdot c)+{sin(x)^{sin({2} \cdot x)^{-(sin({3} \cdot x))}}}) \cdot  \frac {1} {{{cos(x)^{{2}}}}}+sin({5} \cdot x)+cos({x^{{5}}})$\end{center}
 Найдём её производную.\\
\section{ Дифференцирование }
Сложим всё вместе:\\
\begin{center}$(({x^{{2}}}+sinh(a \cdot x \cdot x+b \cdot c)+{sin(x)^{sin({2} \cdot x)^{-(sin({3} \cdot x))}}}) \cdot  \frac {1} {{{cos(x)^{{2}}}}}+sin({5} \cdot x)+cos({x^{{5}}}))'=(({x^{{2}}}+sinh(a \cdot x \cdot x+b \cdot c)+{sin(x)^{sin({2} \cdot x)^{-(sin({3} \cdot x))}}}) \cdot  \frac {1} {{{cos(x)^{{2}}}}})' + (sin({5} \cdot x))' + (cos({x^{{5}}}))'$.\end{center}Derivative of multiplication goes brrr\\
\begin{center}$(({x^{{2}}}+sinh(a \cdot x \cdot x+b \cdot c)+{sin(x)^{sin({2} \cdot x)^{-(sin({3} \cdot x))}}}) \cdot  \frac {1} {{{cos(x)^{{2}}}}})'=({x^{{2}}}+sinh(a \cdot x \cdot x+b \cdot c)+{sin(x)^{sin({2} \cdot x)^{-(sin({3} \cdot x))}}}) \cdot  \frac {1} {{{cos(x)^{{2}}}}}\cdot (Log({x^{{2}}}+sinh(a \cdot x \cdot x+b \cdot c)+{sin(x)^{sin({2} \cdot x)^{-(sin({3} \cdot x))}}})' + log( \frac {1} {{{cos(x)^{{2}}}}})')$.\end{center}
H-H Чтож, если ты хочешь нажать "Мне нравится", тогда спускайся ..\\
H-H\\
H-H\\
H-H\\
\begin{center}$(log({x^{{2}}}+sinh(a \cdot x \cdot x+b \cdot c)+{sin(x)^{sin({2} \cdot x)^{-(sin({3} \cdot x))}}}))'= \frac { ({x^{{2}}}+sinh(a \cdot x \cdot x+b \cdot c)+{sin(x)^{sin({2} \cdot x)^{-(sin({3} \cdot x))}}})'} {{x^{{2}}}+sinh(a \cdot x \cdot x+b \cdot c)+{sin(x)^{sin({2} \cdot x)^{-(sin({3} \cdot x))}}}}$.\end{center}
Сложим всё вместе:\\
\begin{center}$({x^{{2}}}+sinh(a \cdot x \cdot x+b \cdot c)+{sin(x)^{sin({2} \cdot x)^{-(sin({3} \cdot x))}}})'=({x^{{2}}})' + (sinh(a \cdot x \cdot x+b \cdot c))' + ({sin(x)^{sin({2} \cdot x)^{-(sin({3} \cdot x))}}})'$.\end{center}Единица - и всё. Всё пропало.\\
\begin{center}$({x^{{2}}})'={x^{{2}}} \cdot (({{2}})'\cdot log(x) + {{2}} \cdot (logx)')$.\end{center}
Первая строка таблицы производных гласит:\\
\begin{center}$(x)' = 1$.\end{center}
Творческое задание: доказать справедливость формулы, проинтегрировав правую её часть:\\
\begin{center}$({2})' = 0$.\end{center}
Единица - и всё. Всё пропало.\\
\begin{center}$({x^{{2}}})'= {x^{{2}}} \cdot ({1} \cdot  \frac {1} {{x}} \cdot {{2}}+log(x) \cdot {0})$.\end{center}
Если там $log|cosh(x)|$, то я не знаю, что это такое\\
\begin{center}$(sinh(a \cdot x \cdot x+b \cdot c))'= cosh(a \cdot x \cdot x+b \cdot c)$.\end{center}
Не существует такой эпсилон-окрестности этого шага, в которой нет гадости:\\
\begin{center}$(a \cdot x \cdot x+b \cdot c)'=(a \cdot x \cdot x)' + (b \cdot c)'$.\end{center}Умножение, это, с одной стороны, весело, но отладка...\\
\begin{center}$(a \cdot x \cdot x)'=a \cdot x \cdot x\cdot (Log(a)' + Log(x)' + log(x)')$.\end{center}
Ну и очередной логарифм:\\
\begin{center}$(log(a))'= \frac { (a)'} {a}$.\end{center}
Любопытно отметить, что верно данное тождество:\\
\begin{center}$(a)' = 0$.\end{center}
Не каждому дано познать производную логарифма:\\
\begin{center}$(log(a))'=  \frac {1} {{a}} \cdot {0}$.\end{center}
H-H Чтож, если ты хочешь нажать "Мне нравится", тогда спускайся ..\\
H-H\\
H-H\\
H-H\\
\begin{center}$(log(x))'= \frac { (x)'} {x}$.\end{center}
Любопытно отметить, что верно данное тождество:\\
\begin{center}$(x)' = 1$.\end{center}
Каждый комментарий со словом "Полторашка" увеличивает оценку за прогу на 1 балл:\\
\begin{center}$(log(x))'=  \frac {1} {{x}} \cdot {1}$.\end{center}
Творческое задание: доказать справедливость формулы, проинтегрировав правую её часть:\\
\begin{center}$(log(x))'= \frac { (x)'} {x}$.\end{center}
Внимание, анекдот! Купил мужик шляпу, а она ему как раз!\\
\begin{center}$(x)' = 1$.\end{center}
Внимание, анекдот! Купил мужик шляпу, а она ему как раз!\\
\begin{center}$(log(x))'=  \frac {1} {{x}} \cdot {1}$.\end{center}
Не показывайте это кафедре вышмата!!\\
\begin{center}$(a \cdot x \cdot x)'= ( \frac {1} {{a}} \cdot {0}+ \frac {1} {{x}} \cdot {1}+ \frac {1} {{x}} \cdot {1}) \cdot a \cdot x \cdot x$.\end{center}
Не показывайте это кафедре вышмата!!\\
\begin{center}$(b \cdot c)'=b \cdot c\cdot (Log(b)' + log(c)')$.\end{center}
Не хотите купить книжку про эфир?\\
\begin{center}$(log(b))'= \frac { (b)'} {b}$.\end{center}
Здесь совсем всё легко.\\
\begin{center}$(b)' = 0$.\end{center}
щывывщафзывщпыхвыпщпуцщауощршпцпушумншпяюгщшпшпгшиташндзхвыаф:\\
\begin{center}$(log(b))'=  \frac {1} {{b}} \cdot {0}$.\end{center}
H-H Не надоело еще?\\
H-H\\
H-H\\
H-H\\
\begin{center}$(log(c))'= \frac { (c)'} {c}$.\end{center}
А Дуня разливает чай, А Петербург неугомонный, ...\\
\begin{center}$(c)' = 0$.\end{center}
Здесь ограничимся магическим комментарием "Очевидно, что:"\\
\begin{center}$(log(c))'=  \frac {1} {{c}} \cdot {0}$.\end{center}
Derivative of multiplication goes brrr\\
\begin{center}$(b \cdot c)'= ( \frac {1} {{b}} \cdot {0}+ \frac {1} {{c}} \cdot {0}) \cdot b \cdot c$.\end{center}
Сложим всё вместе:\\
\begin{center}$(a \cdot x \cdot x+b \cdot c)'= ( \frac {1} {{a}} \cdot {0}+ \frac {1} {{x}} \cdot {1}+ \frac {1} {{x}} \cdot {1}) \cdot a \cdot x \cdot x+( \frac {1} {{b}} \cdot {0}+ \frac {1} {{c}} \cdot {0}) \cdot b \cdot c$.\end{center}
Так, вроде выучил... Чёрт, так нужен минус или нет?\\
\begin{center}$(sinh(a \cdot x \cdot x+b \cdot c))'= cosh(a \cdot x \cdot x+b \cdot c) \cdot (( \frac {1} {{a}} \cdot {0}+ \frac {1} {{x}} \cdot {1}+ \frac {1} {{x}} \cdot {1}) \cdot a \cdot x \cdot x+( \frac {1} {{b}} \cdot {0}+ \frac {1} {{c}} \cdot {0}) \cdot b \cdot c)$.\end{center}
Здесь нам приходит на помощь логарифм.\\
\begin{center}$({sin(x)^{sin({2} \cdot x)^{-(sin({3} \cdot x))}}})'={sin(x)^{sin({2} \cdot x)^{-(sin({3} \cdot x))}}} \cdot (( \frac {1} {{sin({2} \cdot x)} ^ {sin({3} \cdot x)}})'\cdot log(sin(x)) +  \frac {1} {{sin({2} \cdot x)} ^ {sin({3} \cdot x)}} \cdot (logsin(x))')$.\end{center}
*подглядывает в таблицу производных*\\
\begin{center}$(sin(x))'= cos(x)$.\end{center}
Любопытно отметить, что верно данное тождество:\\
\begin{center}$(x)' = 1$.\end{center}
H-H Чтож, если ты хочешь нажать "Мне нравится", тогда спускайся ..\\
H-H\\
H-H\\
H-H\\
\begin{center}$(sin(x))'= cos(x) \cdot {1}$.\end{center}
Ландау такое в 14 лет мог посчитать:\\
\begin{center}$({sin(x)^{sin({2} \cdot x)^{-(sin({3} \cdot x))}}})'={sin(x)^{sin({2} \cdot x)^{-(sin({3} \cdot x))}}} \cdot (( \frac {1} {{sin({2} \cdot x)} ^ {sin({3} \cdot x)}})'\cdot log(sin(x)) +  \frac {1} {{sin({2} \cdot x)} ^ {sin({3} \cdot x)}} \cdot (logsin(x))')$.\end{center}
*подглядывает в таблицу производных*\\
\begin{center}$(sin({2} \cdot x))'= cos({2} \cdot x)$.\end{center}
H-H Не надоело еще?\\
H-H\\
H-H\\
H-H\\
\begin{center}$({2} \cdot x)'={2} \cdot x\cdot (Log({2})' + log(x)')$.\end{center}
Под бурные аплодисменты, логарифм покидает нас.\\
\begin{center}$(log({2}))'= \frac { ({2})'} {{2}}$.\end{center}
Путём несложных математических преобразований получаем\\
\begin{center}$({2})' = 0$.\end{center}
Кто не верит, пусть загонит в Wolfram\\
\begin{center}$(log({2}))'=  \frac {1} {{{2}}} \cdot {0}$.\end{center}
Не каждому дано познать производную логарифма:\\
\begin{center}$(log(x))'= \frac { (x)'} {x}$.\end{center}
Первая строка таблицы производных гласит:\\
\begin{center}$(x)' = 1$.\end{center}
щывывщафзывщпыхвыпщпуцщауощршпцпушумншпяюгщшпшпгшиташндзхвыаф:\\
\begin{center}$(log(x))'=  \frac {1} {{x}} \cdot {1}$.\end{center}
Derivative of multiplication goes brrr\\
\begin{center}$({2} \cdot x)'= ( \frac {1} {{{2}}} \cdot {0}+ \frac {1} {{x}} \cdot {1}) \cdot {2} \cdot x$.\end{center}
Внимание, анекдот! Купил мужик шляпу, а она ему как раз!\\
\begin{center}$(sin({2} \cdot x))'= cos({2} \cdot x) \cdot ( \frac {1} {{{2}}} \cdot {0}+ \frac {1} {{x}} \cdot {1}) \cdot {2} \cdot x$.\end{center}
Говорят, один программист не признавал унарный минус и ему приходилось дифференцировать произведение (-1) на это выражение\\
\begin{center}$(-(sin({3} \cdot x)))'= -(sin({3} \cdot x))$.\end{center}
После этих слов в украинском поезде начался сущий кошмар:\\
\begin{center}$(sin({3} \cdot x))'= cos({3} \cdot x)$.\end{center}
Двести тысяч логарифмов продифференцировано и ещё миллион на подходе\\
\begin{center}$({3} \cdot x)'={3} \cdot x\cdot (Log({3})' + log(x)')$.\end{center}
Не существует такой эпсилон-окрестности этого шага, в которой нет гадости:\\
\begin{center}$(log({3}))'= \frac { ({3})'} {{3}}$.\end{center}
Здесь совсем всё легко.\\
\begin{center}$({3})' = 0$.\end{center}
H-H Чтож, если ты хочешь нажать "Мне нравится", тогда спускайся ..\\
H-H\\
H-H\\
H-H\\
\begin{center}$(log({3}))'=  \frac {1} {{{3}}} \cdot {0}$.\end{center}
Под бурные аплодисменты, логарифм покидает нас.\\
\begin{center}$(log(x))'= \frac { (x)'} {x}$.\end{center}
Первая строка таблицы производных гласит:\\
\begin{center}$(x)' = 1$.\end{center}
H-H Не спускайся вниз\\
H-H\\
H-H\\
H-H\\
\begin{center}$(log(x))'=  \frac {1} {{x}} \cdot {1}$.\end{center}
Путём несложных математических преобразований получаем\\
\begin{center}$({3} \cdot x)'= ( \frac {1} {{{3}}} \cdot {0}+ \frac {1} {{x}} \cdot {1}) \cdot {3} \cdot x$.\end{center}
А вы когда-нибудь задумывались, что\\
\begin{center}$(sin({3} \cdot x))'= cos({3} \cdot x) \cdot ( \frac {1} {{{3}}} \cdot {0}+ \frac {1} {{x}} \cdot {1}) \cdot {3} \cdot x$.\end{center}
Вынесем минус за знак производной:\\
\begin{center}$(-(sin({3} \cdot x)))'= -(cos({3} \cdot x) \cdot ( \frac {1} {{{3}}} \cdot {0}+ \frac {1} {{x}} \cdot {1}) \cdot {3} \cdot x)$.\end{center}
Здесь нам приходит на помощь логарифм.\\
\begin{center}$({sin(x)^{sin({2} \cdot x)^{-(sin({3} \cdot x))}}})'= {sin(x)^{sin({2} \cdot x)^{-(sin({3} \cdot x))}}} \cdot (cos(x) \cdot {1} \cdot  \frac {1} {{sin(x)}} \cdot  \frac {1} {{sin({2} \cdot x)} ^ {sin({3} \cdot x)}}+log(sin(x)) \cdot  \frac {1} {{sin({2} \cdot x)} ^ {sin({3} \cdot x)}} \cdot (cos({2} \cdot x) \cdot ( \frac {1} {{{2}}} \cdot {0}+ \frac {1} {{x}} \cdot {1}) \cdot {2} \cdot x \cdot  \frac {1} {{sin({2} \cdot x)}} \cdot {-(sin({3} \cdot x))}+log(sin({2} \cdot x)) \cdot -(cos({3} \cdot x) \cdot ( \frac {1} {{{3}}} \cdot {0}+ \frac {1} {{x}} \cdot {1}) \cdot {3} \cdot x)))$.\end{center}
Ноль, целковый...\\
\begin{center}$({x^{{2}}}+sinh(a \cdot x \cdot x+b \cdot c)+{sin(x)^{sin({2} \cdot x)^{-(sin({3} \cdot x))}}})'= {x^{{2}}} \cdot ({1} \cdot  \frac {1} {{x}} \cdot {{2}}+log(x) \cdot {0})+cosh(a \cdot x \cdot x+b \cdot c) \cdot (( \frac {1} {{a}} \cdot {0}+ \frac {1} {{x}} \cdot {1}+ \frac {1} {{x}} \cdot {1}) \cdot a \cdot x \cdot x+( \frac {1} {{b}} \cdot {0}+ \frac {1} {{c}} \cdot {0}) \cdot b \cdot c)+{sin(x)^{sin({2} \cdot x)^{-(sin({3} \cdot x))}}} \cdot (cos(x) \cdot {1} \cdot  \frac {1} {{sin(x)}} \cdot  \frac {1} {{sin({2} \cdot x)} ^ {sin({3} \cdot x)}}+log(sin(x)) \cdot  \frac {1} {{sin({2} \cdot x)} ^ {sin({3} \cdot x)}} \cdot (cos({2} \cdot x) \cdot ( \frac {1} {{{2}}} \cdot {0}+ \frac {1} {{x}} \cdot {1}) \cdot {2} \cdot x \cdot  \frac {1} {{sin({2} \cdot x)}} \cdot {-(sin({3} \cdot x))}+log(sin({2} \cdot x)) \cdot -(cos({3} \cdot x) \cdot ( \frac {1} {{{3}}} \cdot {0}+ \frac {1} {{x}} \cdot {1}) \cdot {3} \cdot x)))$.\end{center}
Не каждому дано познать производную логарифма:\\
\begin{center}$(log({x^{{2}}}+sinh(a \cdot x \cdot x+b \cdot c)+{sin(x)^{sin({2} \cdot x)^{-(sin({3} \cdot x))}}}))'=  \frac {1} {{{x^{{2}}}+sinh(a \cdot x \cdot x+b \cdot c)+{sin(x)^{sin({2} \cdot x)^{-(sin({3} \cdot x))}}}}} \cdot ({x^{{2}}} \cdot ({1} \cdot  \frac {1} {{x}} \cdot {{2}}+log(x) \cdot {0})+cosh(a \cdot x \cdot x+b \cdot c) \cdot (( \frac {1} {{a}} \cdot {0}+ \frac {1} {{x}} \cdot {1}+ \frac {1} {{x}} \cdot {1}) \cdot a \cdot x \cdot x+( \frac {1} {{b}} \cdot {0}+ \frac {1} {{c}} \cdot {0}) \cdot b \cdot c)+{sin(x)^{sin({2} \cdot x)^{-(sin({3} \cdot x))}}} \cdot (cos(x) \cdot {1} \cdot  \frac {1} {{sin(x)}} \cdot  \frac {1} {{sin({2} \cdot x)} ^ {sin({3} \cdot x)}}+log(sin(x)) \cdot  \frac {1} {{sin({2} \cdot x)} ^ {sin({3} \cdot x)}} \cdot (cos({2} \cdot x) \cdot ( \frac {1} {{{2}}} \cdot {0}+ \frac {1} {{x}} \cdot {1}) \cdot {2} \cdot x \cdot  \frac {1} {{sin({2} \cdot x)}} \cdot {-(sin({3} \cdot x))}+log(sin({2} \cdot x)) \cdot -(cos({3} \cdot x) \cdot ( \frac {1} {{{3}}} \cdot {0}+ \frac {1} {{x}} \cdot {1}) \cdot {3} \cdot x))))$.\end{center}
Ну и очередной логарифм:\\
\begin{center}$(log( \frac {1} {{{cos(x)^{{2}}}}}))'= \frac { ( \frac {1} {{{cos(x)^{{2}}}}})'} { \frac {1} {{{cos(x)^{{2}}}}}}$.\end{center}
Творческое задание: доказать справедливость формулы, проинтегрировав правую её часть:\\
\begin{center}$( \frac {1} {{{cos(x)^{{2}}}}})'= \frac {1} {{{cos(x)^{{2}}}}} \cdot (({{-1}})'\cdot log({cos(x)^{{2}}}) + {{-1}} \cdot (log{cos(x)^{{2}}})')$.\end{center}
Единица - и всё. Всё пропало.\\
\begin{center}$({cos(x)^{{2}}})'={cos(x)^{{2}}} \cdot (({{2}})'\cdot log(cos(x)) + {{2}} \cdot (logcos(x))')$.\end{center}
Вспомнишь .. ортогональное проектирование, а вот и косинус.\\
\begin{center}$(cos(x))'= -sin(x)$.\end{center}
Любопытно отметить, что верно данное тождество:\\
\begin{center}$(x)' = 1$.\end{center}
Вычтем это из $\frac {\pi} {2}$ и сведём задачу к предыдущей, уже решённой:\\
\begin{center}$(cos(x))'= -(sin(x)) \cdot {1}$.\end{center}
Здесь ограничимся магическим комментарием "Очевидно, что:"\\
\begin{center}$({2})' = 0$.\end{center}
Задачка со звёздочкой из учебника 11 класса:\\
\begin{center}$({cos(x)^{{2}}})'= {cos(x)^{{2}}} \cdot (-(sin(x)) \cdot {1} \cdot  \frac {1} {{cos(x)}} \cdot {{2}}+log(cos(x)) \cdot {0})$.\end{center}
Кто будет плохо слушать, те сделают это на листочке!\\
\begin{center}$({-1})' = 0$.\end{center}
Задачка со звёздочкой из учебника 11 класса:\\
\begin{center}$( \frac {1} {{{cos(x)^{{2}}}}})'=  \frac {1} {{{cos(x)^{{2}}}}} \cdot ({cos(x)^{{2}}} \cdot (-(sin(x)) \cdot {1} \cdot  \frac {1} {{cos(x)}} \cdot {{2}}+log(cos(x)) \cdot {0}) \cdot  \frac {1} {{{cos(x)^{{2}}}}} \cdot {{-1}}+log({cos(x)^{{2}}}) \cdot {0})$.\end{center}
Каждый комментарий со словом "Полторашка" увеличивает оценку за прогу на 1 балл:\\
\begin{center}$(log( \frac {1} {{{cos(x)^{{2}}}}}))'=  \frac {1} {{ \frac {1} {{{cos(x)^{{2}}}}}}} \cdot  \frac {1} {{{cos(x)^{{2}}}}} \cdot ({cos(x)^{{2}}} \cdot (-(sin(x)) \cdot {1} \cdot  \frac {1} {{cos(x)}} \cdot {{2}}+log(cos(x)) \cdot {0}) \cdot  \frac {1} {{{cos(x)^{{2}}}}} \cdot {{-1}}+log({cos(x)^{{2}}}) \cdot {0})$.\end{center}
Каждый комментарий со словом "Полторашка" увеличивает оценку за прогу на 1 балл:\\
\begin{center}$(({x^{{2}}}+sinh(a \cdot x \cdot x+b \cdot c)+{sin(x)^{sin({2} \cdot x)^{-(sin({3} \cdot x))}}}) \cdot  \frac {1} {{{cos(x)^{{2}}}}})'= ( \frac {1} {{{x^{{2}}}+sinh(a \cdot x \cdot x+b \cdot c)+{sin(x)^{sin({2} \cdot x)^{-(sin({3} \cdot x))}}}}} \cdot ({x^{{2}}} \cdot ({1} \cdot  \frac {1} {{x}} \cdot {{2}}+log(x) \cdot {0})+cosh(a \cdot x \cdot x+b \cdot c) \cdot (( \frac {1} {{a}} \cdot {0}+ \frac {1} {{x}} \cdot {1}+ \frac {1} {{x}} \cdot {1}) \cdot a \cdot x \cdot x+( \frac {1} {{b}} \cdot {0}+ \frac {1} {{c}} \cdot {0}) \cdot b \cdot c)+{sin(x)^{sin({2} \cdot x)^{-(sin({3} \cdot x))}}} \cdot (cos(x) \cdot {1} \cdot  \frac {1} {{sin(x)}} \cdot  \frac {1} {{sin({2} \cdot x)} ^ {sin({3} \cdot x)}}+log(sin(x)) \cdot  \frac {1} {{sin({2} \cdot x)} ^ {sin({3} \cdot x)}} \cdot (cos({2} \cdot x) \cdot ( \frac {1} {{{2}}} \cdot {0}+ \frac {1} {{x}} \cdot {1}) \cdot {2} \cdot x \cdot  \frac {1} {{sin({2} \cdot x)}} \cdot {-(sin({3} \cdot x))}+log(sin({2} \cdot x)) \cdot -(cos({3} \cdot x) \cdot ( \frac {1} {{{3}}} \cdot {0}+ \frac {1} {{x}} \cdot {1}) \cdot {3} \cdot x))))+ \frac {1} {{ \frac {1} {{{cos(x)^{{2}}}}}}} \cdot  \frac {1} {{{cos(x)^{{2}}}}} \cdot ({cos(x)^{{2}}} \cdot (-(sin(x)) \cdot {1} \cdot  \frac {1} {{cos(x)}} \cdot {{2}}+log(cos(x)) \cdot {0}) \cdot  \frac {1} {{{cos(x)^{{2}}}}} \cdot {{-1}}+log({cos(x)^{{2}}}) \cdot {0})) \cdot ({x^{{2}}}+sinh(a \cdot x \cdot x+b \cdot c)+{sin(x)^{sin({2} \cdot x)^{-(sin({3} \cdot x))}}}) \cdot  \frac {1} {{{cos(x)^{{2}}}}}$.\end{center}
Ну что, покажем ему, как чётным быть?\\
\begin{center}$(sin({5} \cdot x))'= cos({5} \cdot x)$.\end{center}
Это даже моя программа посчитала устно:\\
\begin{center}$({5} \cdot x)'={5} \cdot x\cdot (Log({5})' + log(x)')$.\end{center}
Ну и очередной логарифм:\\
\begin{center}$(log({5}))'= \frac { ({5})'} {{5}}$.\end{center}
Любопытно отметить, что верно данное тождество:\\
\begin{center}$({5})' = 0$.\end{center}
щывывщафзывщпыхвыпщпуцщауощршпцпушумншпяюгщшпшпгшиташндзхвыаф:\\
\begin{center}$(log({5}))'=  \frac {1} {{{5}}} \cdot {0}$.\end{center}
Это даже моя программа посчитала устно:\\
\begin{center}$(log(x))'= \frac { (x)'} {x}$.\end{center}
Здесь совсем всё легко.\\
\begin{center}$(x)' = 1$.\end{center}
H-H Не надоело еще?\\
H-H\\
H-H\\
H-H\\
\begin{center}$(log(x))'=  \frac {1} {{x}} \cdot {1}$.\end{center}
Умножение, оно как сложение, но и как степень, только умножение.\\
\begin{center}$({5} \cdot x)'= ( \frac {1} {{{5}}} \cdot {0}+ \frac {1} {{x}} \cdot {1}) \cdot {5} \cdot x$.\end{center}
Каждый комментарий со словом "Полторашка" увеличивает оценку за прогу на 1 балл:\\
\begin{center}$(sin({5} \cdot x))'= cos({5} \cdot x) \cdot ( \frac {1} {{{5}}} \cdot {0}+ \frac {1} {{x}} \cdot {1}) \cdot {5} \cdot x$.\end{center}
Вычтем это из $\frac {\pi} {2}$ и сведём задачу к предыдущей, уже решённой:\\
\begin{center}$(cos({x^{{5}}}))'= -sin({x^{{5}}})$.\end{center}
H-H Не спускайся вниз\\
H-H\\
H-H\\
H-H\\
\begin{center}$({x^{{5}}})'={x^{{5}}} \cdot (({{5}})'\cdot log(x) + {{5}} \cdot (logx)')$.\end{center}
Первая строка таблицы производных гласит:\\
\begin{center}$(x)' = 1$.\end{center}
Первая строка таблицы производных гласит:\\
\begin{center}$({5})' = 0$.\end{center}
Единица - и всё. Всё пропало.\\
\begin{center}$({x^{{5}}})'= {x^{{5}}} \cdot ({1} \cdot  \frac {1} {{x}} \cdot {{5}}+log(x) \cdot {0})$.\end{center}
Не существует такой эпсилон-окрестности этого шага, в которой нет гадости:\\
\begin{center}$(cos({x^{{5}}}))'= -(sin({x^{{5}}})) \cdot {x^{{5}}} \cdot ({1} \cdot  \frac {1} {{x}} \cdot {{5}}+log(x) \cdot {0})$.\end{center}
Сложим всё вместе:\\
\begin{center}$(({x^{{2}}}+sinh(a \cdot x \cdot x+b \cdot c)+{sin(x)^{sin({2} \cdot x)^{-(sin({3} \cdot x))}}}) \cdot  \frac {1} {{{cos(x)^{{2}}}}}+sin({5} \cdot x)+cos({x^{{5}}}))'= ( \frac {1} {{{x^{{2}}}+sinh(a \cdot x \cdot x+b \cdot c)+{sin(x)^{sin({2} \cdot x)^{-(sin({3} \cdot x))}}}}} \cdot ({x^{{2}}} \cdot ({1} \cdot  \frac {1} {{x}} \cdot {{2}}+log(x) \cdot {0})+cosh(a \cdot x \cdot x+b \cdot c) \cdot (( \frac {1} {{a}} \cdot {0}+ \frac {1} {{x}} \cdot {1}+ \frac {1} {{x}} \cdot {1}) \cdot a \cdot x \cdot x+( \frac {1} {{b}} \cdot {0}+ \frac {1} {{c}} \cdot {0}) \cdot b \cdot c)+{sin(x)^{sin({2} \cdot x)^{-(sin({3} \cdot x))}}} \cdot (cos(x) \cdot {1} \cdot  \frac {1} {{sin(x)}} \cdot  \frac {1} {{sin({2} \cdot x)} ^ {sin({3} \cdot x)}}+log(sin(x)) \cdot  \frac {1} {{sin({2} \cdot x)} ^ {sin({3} \cdot x)}} \cdot (cos({2} \cdot x) \cdot ( \frac {1} {{{2}}} \cdot {0}+ \frac {1} {{x}} \cdot {1}) \cdot {2} \cdot x \cdot  \frac {1} {{sin({2} \cdot x)}} \cdot {-(sin({3} \cdot x))}+log(sin({2} \cdot x)) \cdot -(cos({3} \cdot x) \cdot ( \frac {1} {{{3}}} \cdot {0}+ \frac {1} {{x}} \cdot {1}) \cdot {3} \cdot x))))+ \frac {1} {{ \frac {1} {{{cos(x)^{{2}}}}}}} \cdot  \frac {1} {{{cos(x)^{{2}}}}} \cdot ({cos(x)^{{2}}} \cdot (-(sin(x)) \cdot {1} \cdot  \frac {1} {{cos(x)}} \cdot {{2}}+log(cos(x)) \cdot {0}) \cdot  \frac {1} {{{cos(x)^{{2}}}}} \cdot {{-1}}+log({cos(x)^{{2}}}) \cdot {0})) \cdot ({x^{{2}}}+sinh(a \cdot x \cdot x+b \cdot c)+{sin(x)^{sin({2} \cdot x)^{-(sin({3} \cdot x))}}}) \cdot  \frac {1} {{{cos(x)^{{2}}}}}+cos({5} \cdot x) \cdot ( \frac {1} {{{5}}} \cdot {0}+ \frac {1} {{x}} \cdot {1}) \cdot {5} \cdot x+-(sin({x^{{5}}})) \cdot {x^{{5}}} \cdot ({1} \cdot  \frac {1} {{x}} \cdot {{5}}+log(x) \cdot {0})$.\end{center}
\section{ Анализ результатов }
Многочасовая работа и километры исписанной бумаги позволили нам получить следующие результаты:\\
\begin{center}$(({x^{{2}}}+sinh(a \cdot x \cdot x+b \cdot c)+{sin(x)^{sin({2} \cdot x)^{-(sin({3} \cdot x))}}}) \cdot  \frac {1} {{{cos(x)^{{2}}}}}+sin({5} \cdot x)+cos({x^{{5}}}))'= ( \frac {1} {{{x^{{2}}}+sinh(a \cdot x \cdot x+b \cdot c)+{sin(x)^{sin({2} \cdot x)^{-(sin({3} \cdot x))}}}}} \cdot ({x^{{2}}} \cdot ({1} \cdot  \frac {1} {{x}} \cdot {{2}}+log(x) \cdot {0})+cosh(a \cdot x \cdot x+b \cdot c) \cdot (( \frac {1} {{a}} \cdot {0}+ \frac {1} {{x}} \cdot {1}+ \frac {1} {{x}} \cdot {1}) \cdot a \cdot x \cdot x+( \frac {1} {{b}} \cdot {0}+ \frac {1} {{c}} \cdot {0}) \cdot b \cdot c)+{sin(x)^{sin({2} \cdot x)^{-(sin({3} \cdot x))}}} \cdot (cos(x) \cdot {1} \cdot  \frac {1} {{sin(x)}} \cdot  \frac {1} {{sin({2} \cdot x)} ^ {sin({3} \cdot x)}}+log(sin(x)) \cdot  \frac {1} {{sin({2} \cdot x)} ^ {sin({3} \cdot x)}} \cdot (cos({2} \cdot x) \cdot ( \frac {1} {{{2}}} \cdot {0}+ \frac {1} {{x}} \cdot {1}) \cdot {2} \cdot x \cdot  \frac {1} {{sin({2} \cdot x)}} \cdot {-(sin({3} \cdot x))}+log(sin({2} \cdot x)) \cdot -(cos({3} \cdot x) \cdot ( \frac {1} {{{3}}} \cdot {0}+ \frac {1} {{x}} \cdot {1}) \cdot {3} \cdot x))))+ \frac {1} {{ \frac {1} {{{cos(x)^{{2}}}}}}} \cdot  \frac {1} {{{cos(x)^{{2}}}}} \cdot ({cos(x)^{{2}}} \cdot (-(sin(x)) \cdot {1} \cdot  \frac {1} {{cos(x)}} \cdot {{2}}+log(cos(x)) \cdot {0}) \cdot  \frac {1} {{{cos(x)^{{2}}}}} \cdot {{-1}}+log({cos(x)^{{2}}}) \cdot {0})) \cdot ({x^{{2}}}+sinh(a \cdot x \cdot x+b \cdot c)+{sin(x)^{sin({2} \cdot x)^{-(sin({3} \cdot x))}}}) \cdot  \frac {1} {{{cos(x)^{{2}}}}}+cos({5} \cdot x) \cdot ( \frac {1} {{{5}}} \cdot {0}+ \frac {1} {{x}} \cdot {1}) \cdot {5} \cdot x+-(sin({x^{{5}}})) \cdot {x^{{5}}} \cdot ({1} \cdot  \frac {1} {{x}} \cdot {{5}}+log(x) \cdot {0})$.\end{center}

На первый взгляд формула кажется объёмной. Но её можно упростить:\\
Окончательно имеем:\\
\begin{center}$(({x^{{2}}}+sinh(a \cdot x \cdot x+b \cdot c)+{sin(x)^{sin({2} \cdot x)^{-(sin({3} \cdot x))}}}) \cdot  \frac {1} {{{cos(x)^{{2}}}}}+sin({5} \cdot x)+cos({x^{{5}}}))' = ( \frac {1} {{{x^{{2}}}+sinh(a \cdot x \cdot x+b \cdot c)+{sin(x)^{sin({2} \cdot x)^{-(sin({3} \cdot x))}}}}} \cdot ({x^{{2}}} \cdot {2} \cdot  \frac {1} {{x}}+cosh(a \cdot x \cdot x+b \cdot c) \cdot ( \frac {1} {{x}}+ \frac {1} {{x}}) \cdot a \cdot x \cdot x+{sin(x)^{sin({2} \cdot x)^{-(sin({3} \cdot x))}}} \cdot (cos(x) \cdot  \frac {1} {{sin(x)}} \cdot  \frac {1} {{sin({2} \cdot x)} ^ {sin({3} \cdot x)}}+log(sin(x)) \cdot  \frac {1} {{sin({2} \cdot x)} ^ {sin({3} \cdot x)}} \cdot (cos({2} \cdot x) \cdot  \frac {1} {{x}} \cdot {2} \cdot x \cdot  \frac {1} {{sin({2} \cdot x)}} \cdot -(sin({3} \cdot x))+log(sin({2} \cdot x)) \cdot -(cos({3} \cdot x) \cdot  \frac {1} {{x}} \cdot {3} \cdot x))))+{cos(x)^{{2}^{{-1}^{{-1}}}}} \cdot {cos(x)^{{2}^{{-1}}}} \cdot {cos(x)^{{2}}} \cdot -(sin(x)) \cdot  \frac {1} {{cos(x)}} \cdot {-2} \cdot {cos(x)^{{2}^{{-1}}}}) \cdot ({x^{{2}}}+sinh(a \cdot x \cdot x+b \cdot c)+{sin(x)^{sin({2} \cdot x)^{-(sin({3} \cdot x))}}}) \cdot {cos(x)^{{2}^{{-1}}}}+cos({5} \cdot x) \cdot  \frac {1} {{x}} \cdot {5} \cdot x+-(sin({x^{{5}}})) \cdot {x^{{5}}} \cdot {5} \cdot  \frac {1} {{x}}$.\end{center}
\section{ Источники: }1) Ю.С.Рыбников. Счёт древних русов. НеМ: НеПросвещение, 2015 \\
2) ГДЗ по матеше 5 класс - Н.Я.Виленкин. Сайт: https://gdz.ru/class-5/matematika/vilenkin/\\
3) Кричалки Красноярской летней школы - полное собрание. Красноярск: СФУ, 2020\\
4) Дифференциальное и интегральное исчисления. Функции одной переменной. Л.Н.Знаменская. Где-то в Долгопрудном: в каком-то издательстве, когда-то в прошлом\\
\end{document}